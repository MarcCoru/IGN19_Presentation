\documentclass[%
  aspectratio=169,
  9pt,
  USenglish,
  titlegraphic, % store custom image to .images/titlegraphic
  affiliationintitlepagehead,
  progressbar,
%   affiliation,
]{beamer}

\usetheme{TUM}

\input{preamble.tex}

\setbeamertemplate{blocks}[rounded][shadow=false]

\title{Learning Vegetation Models from Satellite Data}
\subtitle{IGN getting-in-touch}
\author[M. Rußwurm]{Marc Rußwurm}
\institute[TUM]{Technical University of Munich, Germany\\
                Remote Sensing Technology}
\date{July 10th, 2019}

\begin{document}
\begin{frame}[t]
  \titlepage
\end{frame}

{\setbeamercolor{background canvas}{bg=black}
	\begin{frame}[plain]
	
	\vfill
	\Huge\color{white}
	\begin{center}
		\begin{columns}
			\column{.5\textwidth}
			\vspace{7em}
			
			\hfill 
			Earth Observation Data
			\column{.5\textwidth}
			
			\includegraphics[width=5cm]{images/epic1}
			%\includegraphics[width=7cm]{images/fdl}
		\end{columns}
	\end{center}
	
	\vfill
\end{frame}
}



{\setbeamercolor{background canvas}{bg=black}
	\begin{frame}[plain]
	
	\vfill
	\Huge\color{white}
	\begin{center}
		\begin{columns}
			\column{.5\textwidth}
			\vspace{7em}
			
			\hfill 
			Earth Observation Data
			\column{.5\textwidth}
			
			\includegraphics[width=5cm]{images/epic1}
			%\includegraphics[width=7cm]{images/fdl}
		\end{columns}
	\end{center}
	
	\vfill
\end{frame}
}

\begin{frame}
\frametitle{System Earth}

\begin{columns}
	
	\column{.5\textwidth}
	
	{
		%		The Earth is a complex system.
		%		Only some components is observable by 
		%		\begin{itemize}
		%			\item satellite-based or
		%			\item in-situ observations
		%		\end{itemize}
		%		
		
		%	\begin{equation*}\V{y} = f({\M{X}})\end{equation*}
		%	partially observe the complex system Earth
		\textbf{Partially measuring} System Earth
		{\Huge
			\begin{equation*}
			\M{X} = \sat\left({\earth}\right)
			\end{equation*}
		}
		
		\vspace{1em}
		\textbf{knowledge extraction} through pattern recognition and machine learning
		
		{\Huge\begin{equation*}\V{?} = f({\M{X}})\end{equation*}}
	}
	
	\column{.5\textwidth}
	
	\begin{tikzpicture}[xscale=3, yscale=2]
	\node(earth) at (0,0) {\includegraphics[width=7cm]{images/epicw1}};	
	
	\end{tikzpicture}
	
	
\end{columns}

%	\includegraphics[width=5cm]{images/Earth_gravity}
%	\includegraphics[width=5cm]{images/epicw1}
%	\includegraphics[width=5cm]{images/earthnullschool/sealevelpressure}
%	\includegraphics[width=5cm]{images/earthnullschool/misery_index}
%	\includegraphics[width=5cm]{images/earthnullschool/temp}
%	\includegraphics[width=5cm]{images/epicw1}
\end{frame}

%\begin{frame}
%\frametitle{Optical Satellites}
%\begin{columns}
%	\column{.5\textwidth}
%	
%%	
%%	\begin{itemize}[itemsep=.5em]
%%		\item<1-> Sensor measures \textbf{Digital Numbers} $\text{DN}(\lambda)$ for each wavelength $\lambda$. 
%%		\item<2-> \textbf{Digital Numbers} are normalized to \textbf{Radiance} 
%%		$L(\lambda), \left[\frac{W}{\text{sr}m^2}\right]$ by gain and offset calibration.
%%		\item<3-> Radiance is normalized to \textbf{top-of-atmosphere reflectance} $\rho(\lambda)$
%%		%		\item<4-> \textbf{Bottom-of-atmosphere reflectances} are reconstructed using a functional model of the atmosphere.
%%	\end{itemize}
%%	
%	%	Radiance $R_\lambda$ from measured Digital Numbers via calibrated gain $\alpha$ and offset $\beta$
%	%	\begin{equation*}
%	%		L_\lambda = \alpha \text{DN}_\lambda + \beta, \left[\frac{W}{\text{sr}m^1}\right]
%	%	\end{equation*}
%	%	
%	%	top-of-atmosphere reflectance $\rho_\lambda$ as normalized Radiance $R_\lambda$ with solar 
%	%	\begin{equation*}
%	%	\rho_\lambda = \frac{L_\lambda}{\cos(\varphi_\text{sun})}
%	%	\frac{
%	%		\pi d^2
%	%	}
%	%	{
%	%		E_\text{sun}(\lambda)
%	%	}
%	%	\end{equation*}
%	%	
%	%	\vspace{1em}
%	%	
%	%	\begin{itemize}
%	%		\item measured radiance $L(\lambda)$
%	%		\item solar irradiance $E_\text{sun}(\lambda)$
%	%		\item solar zenith angle $\varphi_\text{sun}$
%	%		\item squared Earth-Sun distance $d$ in AU
%	%	\end{itemize}
%	
%	
%	\column{.5\textwidth}
%	
%	
%	\begin{tikzpicture}
%	
%	
%	%	\draw [black,dotted, fill=tumbluelight,domain=110:70] plot ({13*cos(\x)}, {13*sin(\x)-12.8});
%	\draw [fill=tumivory,domain=110:70] plot ({10*cos(\x)}, {10*sin(\x)-10});
%	%	\draw [fill=tumbluelight,domain=110:70] plot ({12*cos(\x)}f, {12*sin(\x)-10});
%	
%	
%	\node(sun) at (-2,2) {\includegraphics[width=10mm]{images/icons/sun}};
%	\node[rotate=130,anchor=center](sat) at (2,2) {\includegraphics[width=10mm]{images/icons/sat2}};
%	
%	\node(px) at ({10*cos(90)}, {10*sin(90)-10.1}){
%		\begin{tikzpicture}[xscale=.5,yscale=.25]
%		\draw[fill=tumbluelight] (0,0) -- (1,0) -- (2,1) -- (1,1) -- (0,0);
%		\end{tikzpicture}
%		%\includegraphics[width=5mm]{images/icons/house}
%	};
%	
%	\draw[-stealth] (sun) -- node[midway,sloped]{\wave} (px);
%	\draw[-stealth] (px) -- node[midway,sloped]{\wave} (sat);
%	
%	\visible<3->{\draw[-stealth] (sun) -- node[midway,sloped]{\wave} (sat);
%		\draw[draw=tumgray] (px) -- node[at end,left]{$\varphi_\text{sun}$} ++(0,1.4); 
%		\draw [draw=tumgray, domain=130:90] plot ({1*cos(\x)}, {1*sin(\x)});
%	}
%	
%	\node[above=.5em of sun]{$E_\text{sun}(\lambda)$};
%	\visible<1>{\node[above=4em of sat]{$DN(\lambda)$};}
%	\visible<2>{\node[above=4em of sat]{$L(\lambda)$};}
%	\visible<3>{\node[above=4em of sat]{$\rho_\text{toa}(\lambda)$};}
%	%	\visible<4>{\node[above=4em of sat]{$\rho_\text{boa}(\lambda)$};}
%	
%	%		\draw[red] (0,0) sin (1,2);
%	
%\end{tikzpicture}
%\end{columns}
%\end{frame}

\begin{frame}
\frametitle{Acquired in regular time intervals}
\framesubtitle{Sentinel 2 Satellite}

%	\includemovie[
%	poster,
%	text={\small(Loading Circle-m-increase3.mp4)}
%	]{6cm}{6cm}{images/s2orbits.mp4}
%	
%	\movie{loaded}{images/s2orbits.avi}

\begin{columns}
\column{.5\textwidth}

\Large
\begin{itemize}
\item<1-> polar sun-synchronous orbit
\item<2-> single orbit circa 100 minutes
\item<3-> revisit same location after 5 days
\item<4-> acquisition stripe of 290km width
\item<5-> 13 spectral bands
\item<6-> ground resolution 10-60m
\item<7-> global coverage and free of charge
\end{itemize}

\column{.5\textwidth}
%		\only<1>{\includegraphics[width=\textwidth]{images/s2orbits/1}}
%		\only<2>{\includegraphics[width=\textwidth]{images/s2orbits/2}}
%		\only<3>{\includegraphics[width=\textwidth]{images/s2orbits/3}}
\only<1>{\includegraphics[width=\textwidth]{images/s2orbits/14}}
\only<2>{\includegraphics[width=\textwidth]{images/s2orbits/19}}
\only<3>{\includegraphics[width=\textwidth]{images/s2orbits/24}}
\only<4>{\includegraphics[width=\textwidth]{images/s2orbits/35}}
\only<5>{\includegraphics[width=\textwidth]{images/s2orbits/36}}
\only<6>{\includegraphics[width=\textwidth]{images/s2orbits/37}}
\only<7>{\includegraphics[width=\textwidth]{images/s2orbits/48}}
\tiny\url{https://www.esa.int/spaceinvideos/Videos/2016/08/Sentinel-2_global_coverage}
\end{columns}

\end{frame}

%
%
%{\setbeamercolor{background canvas}{bg=tumblack}
%	\begin{frame}[plain]
%	
%	\vfill
%	\Huge\color{white}
%	\begin{center}
%		\begin{columns}
%			\column{.5\textwidth}
%			\vspace{7em}
%			
%			\hfill 
%			Satellite Data Take-away
%			\column{.5\textwidth}
%			
%			
%			%%			\includegraphics[width=5cm]{images/epic1}
%			%			\includegraphics[width=7cm]{images/fdl}
%		\end{columns}
%	\end{center}
%	
%	\vfill
%\end{frame}
%}

{\setbeamercolor{background canvas}{bg=tumbluedark}
\begin{frame}[plain]

\vfill
\Huge\color{white}
\begin{center}
\begin{columns}
\column{.5\textwidth}
\vspace{7em}

\hfill 
Vegetation Modeling
\column{.5\textwidth}

\includegraphics[width=\textwidth]{images/Large1954_cerial_growth_stages_white}
%%			\includegraphics[width=5cm]{images/epic1}
%			\includegraphics[width=7cm]{images/fdl}
\end{columns}
\small\raggedleft(Large et al., 1954)
\end{center}

\vfill
\end{frame}
}

%\begin{frame}
%
%\frametitle{Photosynthesis}
%%	
%%	Photosynthesis
%
%\centering
%\begin{tikzpicture}
%\node(in){$6{\text{CO}}_{2}+6{\text{H}}_{2}\text{O}$};
%\node[right=of in, label={light absorbtion $\Delta \V{x}$}](arrow){$\to$};
%\node[right=of arrow]{${\text{C}}_{\text{6}}{\text{H}}_{\text{12}}{\text{O}}_{\text{6}}+{\text{6O}}_{\text{2}}$};
%\end{tikzpicture}
%%	
%%	\begin{equation*}
%%	6{\text{CO}}_{2}+6{\text{H}}_{2}\text{O}\to 
%%	\end{equation*}
%\end{frame}

\newcommand{\rastergrid}{
\begin{tikzpicture}
% each layer
\begin{scope}[scale=2]

% raster size
\def\d{0.7}		

% distance layer
\def\s{\d*5}

\foreach \i in {1,...,6}
{		
\begin{scope}[
yshift=\s*\i,every node/.append style={
yslant=0.5,xslant=-1},yslant=0.5,xslant=-1
]
%\draw[step=3.33mm] (0,0) grid (1,1);
%\fill[black,fill opacity=.9] (0.333,0.333) rectangle (0.333,0.333);    	    	  

\foreach \row in {0,...,2}{
\foreach \col in {0,...,2}{
\draw[tumblack, fill=tumblue!\pdfuniformdeviate 40,fill opacity=1,rounded corners=1] (\col*\d/3,\row*\d/3) rectangle (\col*\d/3+\d/3, \row*\d/3+\d/3);
%                 \draw[black, fill=black!\pdfuniformdeviate 40,fill opacity=1,rounded corners=1] (\col*\d/3,\row*\d/3) rectangle (\col*\d/3+\d/3, \row*\d/3+\d/3);
}
}

%\draw[step=3.33mm] (0,0) grid (1,1);
%\fill[white,fill opacity=.9] (0,0) rectangle (1,1);
\end{scope}
}
\end{scope}
\end{tikzpicture}
}


%\begin{frame}
%\frametitle{Spectral Band}
%\end{frame}


\begin{frame}
\frametitle{Multi-temporal Vegetation Modeling}

\begin{columns}
\column{.5\textwidth}

\begin{tikzpicture}
\node[] at (0,0){\includegraphics[width=\textwidth]{images/Large1954_cerial_growth_stages}};

%		\draw[step=1.0,black,thin, fill=none] (-2,-2) grid (2,2);

\visible<-1>{\draw [fill=white, draw=none, opacity=0.8] (-0.8,-3) rectangle (2,2.5);}
\visible<-2>{\draw [fill=white, draw=none, opacity=0.8] (2,-3) rectangle (5,2.5);}

\visible<1>{\node[rotate=190] at (-2.5,1.5){\includegraphics[width=15mm]{images/icons/sat2}};}
\visible<2>{\node[rotate=225] at (-2.5,1.5){\includegraphics[width=15mm]{images/icons/sat2}};}
\visible<3->{\node[rotate=260] at (-2.5,1.5){\includegraphics[width=15mm]{images/icons/sat2}};}


\visible<4->{\node at (-1.5,1.4) {\includegraphics[width=10mm]{images/cloud}};}

\end{tikzpicture}

\column{.5\textwidth}

{\Large
\only<1>{
\begin{equation*}
f_\text{vegetation}(\V{X}_t)
\end{equation*}
}
\only<2>{
\begin{equation*}
f_\text{vegetation}(\V{X}_t,\V{X}_{t+1})
\end{equation*}
}
\only<3>{
\begin{equation*}
f_\text{vegetation}(\V{X}_t,\V{X}_{t+1},\V{X}_{t+2})
\end{equation*}
}
}


\vspace{2em}


\visible<1->{\includegraphics[width=.22\textwidth]{images/s2grid/1}}
\visible<2->{\includegraphics[width=.22\textwidth]{images/s2grid/2}}
\visible<3->{\includegraphics[width=.22\textwidth]{images/s2grid/3}}
\visible<4->{\includegraphics[width=.22\textwidth]{images/s2grid/4}}

\vspace{1em}

{\small 
Large, E. C. (1954). Growth stages in cereals illustration of the Feekes scale. Plant pathology, 3(4), 128-129.
}


\end{columns}
\end{frame}


\begin{frame}
\frametitle{Problem Definition}
\Large


\centering\begin{tikzpicture}[node distance=0em]
\visible<2->{\node(y){\V{y}};}
\visible<2->{\node[right=of y](equals){$=$};}
\node[right=of equals](f){$f_\text{vegetation}$};
\visible<1->{\node[right=of f](x){$(\V{X}_t,\V{X}_{t+1},\V{X}_{t+2})$};}

\end{tikzpicture}

\vspace{1em}
\raggedright

\begin{description}\setlength\itemsep{1em}
	\item[\color{tumblue}Problem:]<1-> \textbf{un/self-supervised learning} of a vegetation model \textbf{is difficult}
	\item[\color{tumblue}Solution:]<2-> re-framing as \textbf{supervised classification} of crop type labels
	\item[\color{tumblue}Intuition:]<3-> A \textbf{supervised classification model} must \textbf{internalize} a learned \textbf{discriminative model} for the \textbf{vegetation}
\end{description}

\end{frame}


\begin{frame}
\frametitle{Multi-temporal Earth observation}
\centering
\begin{tikzpicture}[scale=2]
%\draw[fill=tumblue, draw=none, opacity=0.5](-1,0) circle (1.5);
\node[fill=tumgraylight, draw=none, opacity=0.5, circle, minimum width=6cm, label=Earth Observation] at (-1,0){};

\node[fill=tumgraylight, draw=none, opacity=0.5, circle, minimum width=6cm, label=Machine Learning] at (1,0){};

\visible<1->{
	\node[font=\bfseries, circle, fill=tumbluelight, text width=2.5cm] (vhr) at (-1.5,.7) {high spatial \\ resolution};
	\node[font=\bfseries, circle, fill=tumorange!50, text width=2.5cm] (cv) at (1.5,.7) {computer vision methods};
	\draw[stealth-stealth, very thick] (vhr) -- node[midway,above]{well established} (cv);
}

\visible<2->{
	\node[font=\bfseries, circle, fill=tumbluelight, text width=2.5cm] (mt) at (-1.5,-.7) {high temporal resolution};
	\node[font=\bfseries, circle, fill=tumorange!50, text width=2.5cm] (nlp) at (1.5,-.7) {natural \\ language \\ processing};
	\draw[stealth-stealth, dotted] (mt) -- node[midway,above]{hardly anyone} (nlp);
}

\visible<3->{
	\node[fit=(nlp)(mt), draw, inner sep=.5em, rounded corners, thick, label=above:{\bfseries \Large my focus}]{};
}
%\draw[-stealth] (cv) -- (0,0);

%\draw[-stealth, very thick] (phd) -- (0,, fil0);
\end{tikzpicture}

\end{frame}

\begin{frame}
\frametitle{Example Analogy to Natural Language Processing}
\input{images/analogy_nlp_eo.tikz}
\end{frame}



%
%\begin{frame}
%\frametitle{Between two fields}
%\centering
%\begin{tikzpicture}[scale=2]
%%\draw[fill=tumblue, draw=none, opacity=0.5](-1,0) circle (1.5);
%\node[fill=tumbluelight, draw=none, opacity=0.5, circle, minimum width=6cm, label=Earth Observation] at (-1,0){};
%
%\node[fill=tumorange!50, draw=none, opacity=0.5, circle, minimum width=6cm, label=Machine Learning] at (1,0){};
%%\draw[fill=tumorange, draw=none, opacity=0.5](1,0) circle (1.5);
%
%\draw[-stealth, shorten >=1cm] (-1.3,1) -- (0,0);
%\draw[-stealth, shorten >=1cm] (-1.6,-1) -- (0,0);
%\draw[-stealth, shorten >=1cm] (-1.2,-.6) -- (0,0);
%\draw[-stealth, shorten >=1cm] (-1.8,.4) -- (0,0);
%\draw[-stealth, shorten >=1cm] (-1.3,.2) -- (0,0);
%
%\node[font=\bfseries] at (-2,0) {applications};
%
%\draw[-stealth, shorten >=1cm] (1.3,1) -- (0,0);
%\draw[-stealth, shorten >=1cm] (1.6,-1) -- (0,0);
%\draw[-stealth, shorten >=1cm] (1.2,-.6) -- (0,0);
%\draw[-stealth, shorten >=1cm] (1.8,.4) -- (0,0);
%\draw[-stealth, shorten >=1cm] (1.3,.2) -- (0,0);
%
%\node[font=\bfseries] at (2,0) {methods};
%
%\node[text width=2cm, circle, fill=tumblue, text=white]{global scale};
%
%\node[text width=2cm, circle, fill=tumblue, text=white]{global scale};
%
%\node[font=\normalsize, fill=white, text width=2cm, rounded corners, fill opacity=.5, text opacity=1](phd) at (0,0){global scaleability \\ real world impact \\ Open Data};
%
%%\draw[-stealth, very thick] (phd) -- (0,, fil0);
%\end{tikzpicture}
%
%
%\end{frame}


%



%
%
%\begin{frame}
%	\frametitle{Natural Language Processing}
%	
%	GPT-2 
%%	\cite{radford2019language}
%	
%	Bert Model Pretraining
%%	\cite{Devlin2018bert}
%	
%	
%\end{frame}



\input{images/input.tikz}


\begin{frame}
	\frametitle{Area of Interest in Bavaria}
	\includegraphics[width=\textwidth]{images/aoi}
\end{frame}



\begin{frame}[t]
\frametitle{Looking at Sequence to Sequence Models from NLP}
%	\framesubtitle{Natürlicher Sprachverarbeitung, Übersetzung, Spracherkennung}

%Verbreitet in sequenziellen Aufgabengebieten, wie natürlicher Sprachverarbeitung, Übersetzung, Spracherkennung
%	\begin{center}
%		\large Wort-Sequenz $\rightarrow$ Representation $c$ $\rightarrow$ Wort-Sequenz
%	\end{center}
\begin{center}
	\input{images/seq2seq.tikz}
\end{center}
%\begin{columns}
%	\column{.4\textwidth}
%	%		\brand{Word2Vec} embedding 
%	%		Neural Machine Translation encodes 
%	
%	\centering 
%%	
%%	\begin{tikzpicture}
%%	\node[fill=tumbluelight!50,rounded corners](x){Eingabesequenz};
%%	\node[fill=tumbluelight!50,rounded corners, below=of x](c){Repräsentation};
%%	\node[fill=tumbluelight!50,rounded corners, below=of c](o){Ausgabsequenz};
%%	
%%	\draw[-stealth] (x) -- (c);
%%	\draw[-stealth] (c) -- (o);
%%	
%%	\end{tikzpicture}
%%	
%	\column{.6\textwidth}
%	
%\end{columns}


{\small
	Sutskever, I., Vinyals, O., \& Le, Q. V. (2014). Sequence to sequence learning with neural networks. In Advances in neural information processing systems (pp. 3104-3112).}


\end{frame}

\input{images/encoder.tikz}
\begin{frame}
\frametitle{Temporal Vegetation Modelling with LSTMs}
\framesubtitle{CVPR Earthvision 2017}
\begin{columns}
	\column{.5\textwidth}
	\figencoderfieldRNN
	\column{.5\textwidth}
	\includegraphics[width=.8\textwidth]{images/confmat_fieldrnn}
%		images/stmelfadditional/Genauigkeiten/accuracies1}
\end{columns}

\vspace{1em}

\textsl{\small
	Rußwurm, M. and Körner, M. (2017). \textbf{Temporal Vegetation Modelling using Long Short-Term Memory Networks for Crop Identification from Medium-Resolution Multi-Spectral Satellite Images}. In IEEE/ISPRS EarthVision 2017 Workshop, Proceedings of the IEEE CVPR Workshops. \textbf{\color{tumorange} Best Paper Award}
}

\end{frame}



\def\fps{3}
\input{images/cells.tikz}

\begin{frame}[t]
	\frametitle{Convolutional Long-Short Term Memory Cells (ConvLSTM)}
	
	%	Reveal Sequence:
	%	1: update, empty cell
	%	2: forget gate
	%	3: input gate + modulation gate + jmult
	%	4: output gate
	%	5: cell state
	%	6: output
	
	\begin{columns}
		\begin{column}{0.3\textwidth}
			\begin{tabularx}{\textwidth}{l}
				LSTM update: \\
				$\VHidden_t,\VCellState_t  \leftarrow \VInput_t, \VHidden_{t-1}, \VCellState_{t-1}$\\
				\\
				\visible<2->{Internal gates:} \\
				\visible<2->{$\VForgetGate_t = \sigma( \conv{\concat{ \VInput_t }{ \VHidden_{t-1} } }{ \MWeight_f } + 1 )$} \\
				\visible<3->{$\VInputGate_t = \sigma( \conv{ \concat{ \VInput_t }{ \VHidden_{t-1} } }{ \MWeight_i } ) $} \\
				\visible<3->{$\VModulationGate_t = \sigma( \conv{ \concat{ \VInput_t }{ \VHidden_{t-1} } }{ \MWeight_j } ) $} \\
				\visible<4->{$\VOutputGate_t = \sigma( \conv{ \concat{ \VInput_t }{ \VHidden_{t-1} } }{ \MWeight_o} ) $} \\
				\\
				\visible<5->{Internal cell state:} \\
				\visible<5->{$\VCellState_t = \VCellState_{t-1} \odot \VForgetGate_t + \VInputGate_t \odot \VModulationGate_t$} \\
				\visible<6->{Output:} \\
				\visible<6->{$\VHidden_t= \VOutputGate_t \odot \tanh(\VCellState_t) $} \\
			\end{tabularx}
		\end{column}
		\begin{column}{0.7\textwidth}
			\lstmexplain
		\end{column}
	\end{columns}
\end{frame}


%\input{images/encoder.tikz}


%\begin{frame}
%\frametitle{Classification ConvLSTM Network}
%\input{images/network.tikz}
%\end{frame}


\begin{frame}[t]
\frametitle{Recurrent Convolutional Neural Networks}
\framesubtitle{Convolutional Long Short-Term Memory --- ConvLSTM (Hochreiter \& Schmidhuber, 1997)}
\begin{columns}
	\begin{column}{0.3\textwidth}
		
		
		\textbf{Convolutional LSTM:}
		
		\vspace{2em}
		{\scriptsize
			Xingjian, S. H. I., Chen, Z., Wang, H., Yeung, D. Y., Wong, W. K., \& Woo, W. C. (2015). Convolutional LSTM network: A machine learning approach for precipitation nowcasting. In Advances in neural information processing systems (pp. 802-810). \par
		}
		
	\end{column}
	\vspace{-2em}	
	\begin{column}{0.7\textwidth}
		
		\includegraphics[width=\textwidth]{images/convlstm}	
		%\lstmexplain
	\end{column}
\end{columns}



\end{frame}

\begin{frame}
\frametitle{Classification ConvLSTM Network}

\centering\includegraphics[width=.8\textwidth]{images/lstm}

	\small
\textsl{
	Rußwurm M., Körner M. (2018). \textbf{Multi-Temporal Land Cover Classification with Sequential Recurrent Encoders}. In ISPRS International Journal of Geo-Information. 
}

\end{frame}


%{
%	\setbeamercolor{background canvas}{bg=tumbluedark}

\begin{frame}

\vfill
\Huge\color{black}
\begin{center}
	\begin{columns}
		\column{\textwidth}
		
		
		\includegraphics[width=\textwidth]{images/x_1}
		\includegraphics[width=\textwidth]{images/x_2}
		\includegraphics[width=\textwidth]{images/x_3}
		
		\vspace{3em}
		
		
		\textbf{\color{tumbluedark}Surprise:} 
		\hfill It worked without specifically labeling clouds!
%		\only<1>{Surprise:} \only<2>{It worked without labeling clouds!}
		%			\includegraphics[width=5cm]{images/dscovrepic/epic1}
		%\includegraphics[width=7cm]{images/fdl}
	\end{columns}
\end{center}

\vfill

\end{frame}
%}

\begin{frame}
\frametitle{ConvLSTM robust to clouds}
\input{images/scl.tikz}
\input{images/clouds.tikz}
\end{frame}


\begin{frame}
	\frametitle{Remembering Karpathy's "Unreasonable Effectiveness of RNNs"}
	
	\includegraphics[width=\textwidth]{images/karpathy}
	
	\url{http://karpathy.github.io/2015/05/21/rnn-effectiveness/}
\end{frame}

\input{images/activations.tikz}
\begin{frame}
\frametitle{Internal States Encode increasingly Classification Features}
\framesubtitle{LSTM cell \textbf{47} of 256}
\figactivations{1}{3}
\end{frame}
%

\begin{frame}
\frametitle{Found Cloud Masking Cells in the RNN}
\framesubtitle{LSTM cell \textbf{47} of 256}
\figactivations{1}{22}
\end{frame}

\begin{frame}
\frametitle{Found Cloud Masking Cells in the RNN}
\framesubtitle{LSTM cell \textbf{47} of 256}
\figactivations{1}{47}
\end{frame}
%
%\begin{frame}
%	\frametitle{Publication and Github}
%	
%			Including the spatial information by adding convolutions
%	\vspace{2em}
%	{\small
%		Rußwurm, M., \& Körner, M. (2018). Multi-temporal land cover classification with sequential recurrent encoders. ISPRS International Journal of Geo-Information, 7(4), 129.
%	}
%
%	Github
%	\url{https://github.com/tum-lmf/mtlcc}
%	
%\end{frame}


\begin{frame}[c]
\frametitle{Paper and Code}
\centering 

%	\vspace{3em}

\large



Github + DockerHub + Continuation with GAF AG

\vspace{1ex}

\includegraphics[width=2cm]{images/github} \hspace{.5ex}
\includegraphics[width=2cm]{images/qr_github} \hspace{.5ex}
\includegraphics[width=2cm]{images/docker} \hspace{.5ex}
\vline
\hspace{.5ex}
\includegraphics[width=4cm]{images/gaf}

\vspace{1ex}

\url{https://github.com/TUM-LMF/MTLCC}
\url{https://github.com/TUM-LMF/MTLCC-pytorch}

\url{http://www.lmf.bgu.tum.de/vision/}

\vspace{1em}
\begin{columns}[t]
	\column{.5\textwidth}
	\scriptsize
	\textsl{
		Rußwurm, M. and Körner, M. (2017). \textbf{Temporal Vegetation Modelling using Long Short-Term Memory Networks for Crop Identification from Medium-Resolution Multi-Spectral Satellite Images}. In IEEE/ISPRS EarthVision 2017 Workshop, Proceedings of the IEEE CVPR Workshops.
	}
	
	\column{.5\textwidth}
	\small
	\textsl{
		Rußwurm M., Körner M. (2018). \textbf{Multi-Temporal Land Cover Classification with Sequential Recurrent Encoders}. ISPRS International Journal of Geo-Information. https://arxiv.org/abs/1802.02080. (in review)
	}
	
\end{columns}


\end{frame}

%\begin{frame}
%\frametitle{Found Cloud Masking Cells in the RNN}
%\framesubtitle{LSTM cell \textbf{47} of 256}
%\figactivations{1}{3}
%\end{frame}


\input{sections/earlyclassification}

\begin{frame}
	\frametitle{Research Agenda}
	\framesubtitle{Supervision}
	
	\begin{tikzpicture}[xscale=1]
		\draw[-Stealth]  (0,0) node[left, at end]{strong} -- node[right, at end]{weak} (10,0) ;
		\draw (1,-1) -- (1,1) node[above, at end]{(Conv)RNN};
		\draw (4,-1) -- (4,1) node[above, at end]{Early Classification};
		\draw (7,-1) -- (7,1);
		
	\end{tikzpicture}
	
	\begin{itemize}
		\item away from strict supervised learning
		\item large scale dataset creation
		\item domain transfer between regions
		\item vegetation modeling
	\end{itemize}
\end{frame}

\begin{frame}

\frametitle{Research Agenda}
\framesubtitle{Generalization}

\begin{columns}
	\column{.5\textwidth}
	
	\includegraphics[width=.9\textwidth]{images/France}
	
	\column{.5\textwidth}
	
	\includegraphics[width=.8\textwidth]{images/Bavaria}
\end{columns}

\end{frame}

\begin{frame}
\frametitle{Crop Type Labels in Europe}

\begin{columns}
	
	\column{.5\textwidth}
	
	\Large
	
	\begin{description}\setlength\itemsep{1em}
		\item[\color{tumblue}collected] yearly within European \textbf{Common Agricultural Policy} (CAP)
		\item[\color{tumblue}declared] by Farmers at \textbf{crop subsidy} applications
		\item[\color{tumblue}today] slowly made publicly available (on a national basis)
		\item[\color{tumblue}in future] further harmonized within \textbf{Europe's INSPIRE} directive
	\end{description}
	
	\column{.5\textwidth}
	\includegraphics[width=\textwidth]{images/europe_data2}
	
	
\end{columns}
\end{frame}


{\setbeamercolor{background canvas}{bg=tumbluedark}
	\begin{frame}[plain]
	
	\vfill
	\Huge\color{white}
	\begin{center}
		\begin{columns}
			\column{.5\textwidth}
			\vspace{7em}
			
			\hfill 
			The Dataset
			\column{.5\textwidth}
			
			\includegraphics[width=\textwidth]{images/map/breizh_white}
			%%			\includegraphics[width=5cm]{images/epic1}
			%			\includegraphics[width=7cm]{images/fdl}
		\end{columns}
		\small\raggedleft Field Parcel Coverage in Brittany
	\end{center}
	
	\vfill
\end{frame}
}

\begin{frame}
\frametitle{Organization}

\begin{columns}
\column{.5\textwidth}
\only<1>{
	
	
	\emph{Nomenclature des unités territoriales statistiques (NUTS)} as European standard of administrative Boundaries. 
	\begin{description}
		\item[NUTS-0] countries
		\item[NUTS-1] states
		\item[NUTS-2] districts
		\item[NUTS-3] municipalities
	\end{description}%NUTS-0 countries, NUTS-1 states, NUTS-2 districts, NUTS-3 municipalities. Add-on: All European Statistics (Eurostats) are collected in these administrative divisions.
	
}
\only<2>{
	Field Parcels
	\begin{tabular}{lrrr}
		\toprule
		Departements & NUTS-3 & Parcels & Size \\
		\cmidrule(lr){1-1}\cmidrule(lr){2-2}\cmidrule(lr){3-3}\cmidrule(lr){4-4}
		%			\midrule
		\color{frh04color}\textbf{Morbihan} & FRH04 & 158522 & 4.3 Gb\\
		\color{tumgraydark}\textbf{Côtes-d’Armor} & FRH01 & 221095  & 6.7 Gb \\
		\color{frh02color}\textbf{Finistère} & FRH02 & 180565 & 6.2 Gb \\
		\color{frh03color}\textbf{Ille-et-Vilaine} & FRH03 & 207993 & 6.8 Gb\\
		\\
		\midrule
		Brittany & FRH0 & 768175\\
		\bottomrule
	\end{tabular}
}
\vspace{1em}

we suggest a spatially distinct train/test split via these partitions. \\
\vspace{1em}
\textbf{Bonus:} Eurostat statistics are gathered on NUTS boundaries. 

\column{.5\textwidth}
\includegraphics[width=.9\textwidth]{images/map/regions}
NUTS-3 regions within Brittany, France

\end{columns}

\end{frame}


\input{images/example_brc.tikz}
\begin{frame}
\frametitle{Corn Example}
\examplecorn
\end{frame}

%
%\begin{frame}
%\frametitle{Examples}
%\centering
%%\externalize{beispiel}{%
%\begin{tikzpicture}[node distance=.1em]
%%	\visible<1>{
%%	\node[font=\huge](veg) at (0,0){Vegetationsmodell};
%%	\node[right=0em of veg](input){$\left(\satimage{1},\satimage{1},\satimage{3}\right)$};
%%	}
%\node[font=\huge, label=$f$](equals){$\leftarrow$};
%%	\visible<2>{\node[right=0em of equals](input){\timeseries{input.csv}};}
%\visible<1>{
%	\node[right=0em of equals](input){\timeseries{prep74526670.csv}};}
%
%\visible<2>{\node[right=0em of equals](input){\timeseries{prep77770412.csv}};}
%
%\visible<1>{\node[left= of equals, font=\huge](probas){meadows};}
%\visible<2>{\node[left= of equals, font=\huge](probas){corn};}
%
%
%\end{tikzpicture}%
%%}
%\end{frame}


{\setbeamercolor{background canvas}{bg=tumbluedark}
\begin{frame}[plain]

\vfill
\Huge\color{white}
\begin{center}
\begin{columns}
\column{.5\textwidth}
\vspace{7em}

\hfill 
Baseline Results
\column{.5\textwidth}

%			\includegraphics[width=5cm]{images/epic1}
%\includegraphics[width=7cm]{images/fdl}
\end{columns}
\end{center}

\vfill
\end{frame}
}

\input{images/confmat.tikz}

\begin{frame}
\frametitle{Two Baseline Models}

\Large
Inspired by Models used in NLP, we implemented a \textbf{multi-layer LSTM} and a \textbf{(minified) Transformer encoder}.

\vspace{1em}
\normalsize

%	\begin{table*}[b]
%		\caption{Accuracy metrics for the Multi-layer bidirectional LSTM \cite{hochreiter1997long} and the Transformer-Encoder \cite{Vaswani:transformer}.}
%		\label{tab:accuracies}
%		\centering
\begin{tabular}{lrrrrrrr}
\toprule
baseline & accuracy & $\kappa$ & mean f1 & mean precision  & mean recall \\
\cmidrule(lr){1-1}\cmidrule(lr){2-2}\cmidrule(lr){3-3}\cmidrule(lr){4-4}\cmidrule(lr){5-5}\cmidrule(lr){6-6}\cmidrule(lr){7-7}
Transformer {\small (Vaswani et al., 2017)} & \textbf{0.69}  &  \textbf{0.63} & 0.57 & {0.60} & 0.56 \\
LSTM {\small (Hochreiter and Schmidhuber, 1997)} & 0.68 & 0.62 & \textbf{0.59} & \textbf{0.63} & \textbf{0.58} \\
\bottomrule
\end{tabular}

\vspace{1em}

\Large
\textbf{Takeaway:} 
\begin{itemize}
\item Models perform quite similar
\item Potential for improvement
\item well-defined classes accurately classified
\item broadly defined classes systematically confused
\end{itemize}
%	\end{table*}

\end{frame}

%\begin{frame}
%\frametitle{Results}
%\framesubtitle{LSTM Model}
%\small
%\begin{columns}
%\column{.5\textwidth}
%
%\begin{tabular}{rlcccc}
%\toprule
%\textbf{\#} & \textbf{crop type} &  \textbf{prec.} & \textbf{rec.} & \textbf{$f_1$} & \textbf{\#samples} \\
%\cmidrule(lr){1-1}\cmidrule(lr){2-2}\cmidrule(lr){3-3}\cmidrule(lr){4-4}\cmidrule(lr){5-5}\cmidrule(lr){6-6}
%%\midrule
%1 & barley &         90 &          86 &          88 &     4982 \\
%2 & wheat &         83 &          95 &          89 &    13850 \\
%3 & corn &         93 &          \textbf{96} &          94 &    25059 \\
%4 & fodder &         51 &          34 &          41 &     3449 \\
%5 & fallow &         30 &           2 &           4 &     3863 \\
%6 & misc. &         50 &          49 &          49 &    12499 \\
%7 & orchards &         21 &           7 &          10 &      391 \\
%8 & cereals &         74 &          47 &          57 &     4645 \\
%9 & perm. meadows &         51 &          47 &          49 &    20966 \\
%10 & protein crops &         42 &          61 &          50 &      498 \\
%11 & rapeseed &         \textbf{96} &          94 &          \textbf{95} &     2664 \\
%12 & temp. meadows &         56 &          68 &          62 &    29977 \\
%13 & vegetables &         86 &          69 &          76 &     3114 \\
%&                       &            &             &             &          \\
%&                       &         \textbf{63} &          \textbf{58} &          \textbf{59} &   125957 \\
%\bottomrule
%\end{tabular}
%
%
%\column{.5\textwidth}
%
%\confmat{images/data/BreizhCrops_rnn/npy/confmat_flat.csv}{3}{1}
%\end{columns}
%\end{frame}

\begin{frame}
\frametitle{Summary}

\Large

\begin{columns}[t]

\column{.5\textwidth}

\visible<1->{
%	\textbf{Impact}
%	\vspace{1em}
%	
%	\begin{description}[itemsep=.5em]
%		\item large scale \textbf{real-world dataset}
%		\item effectively \textbf{unlimited data} (spatially and temporally)
%		\item \textbf{assessing generalization} over large regions
%		\item potential for further \textbf{vegetation characteristics} (drought indicator, early classification, crop yield)
%	\end{description}
\textbf{Summary}
\vspace{1em}

\begin{itemize}[itemsep=.5em]
\item<1-> we gathered, compiled, harmonized a \textbf{large supervised classification dataset} (20 Gb of data) for crop type mapping
\item<2-> we \textbf{prove the feasibility} of classification with two deep neural network classifiers (LSTM and Transformer)
\item<3-> baslines \textbf{leave potential} \textbf{for improvement} by future research
\end{itemize}


}

\column{.5\textwidth}

\textbf{Challenges}
\vspace{1em}

\begin{description}[itemsep=.5em]
\item<4-> \textbf{Imbalanced} class \textbf{labels}
\item<5-> Classes with \textbf{similar characteristics}
\item<6-> Non-Gaussian noise induced by \textbf{clouds}
\item<7-> \textbf{Regional} \textbf{variations} in the class distributions
\item<8-> \textbf{Spatial} \textbf{autocorrelation}
\item<9-> \textbf{Irregular} temporal \textbf{sampling} distance
\item<10-> \textbf{Variable} \textbf{sequence} length
\end{description}



\end{columns}

\end{frame}

\end{document}